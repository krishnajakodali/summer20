The multipath fading cahnnel is modelled as a linear finite impulse-reponse filter.
\\
Let $s_i$ denote the set of samples at the input to the channel, Then samples $Rk_i$ at the output of the channel are related to $s_i$ through:
\begin{align}
Rk_i=\sum_{n=-N_1}^{N2}s_{i-n}g_n
\\
Rk=Ak \circledast g
\end{align}
Where $g_n$ is the set of tap weights given by:
\begin{align}
g_n = \sum_{k=1}^{k}a_k \sinc{\brak{\frac{\tau_k}{t_s}-n}}
\\
-N_1 \leq n \leq N_2
\end{align}
In the equations:
\\$t_s$ is the input sample period to the channel
\\$\tau_k$ where $1\leq k \leq K$is the set of path delays(pd).
\\K is the total number of paths in the multiple fading channel.Here, K=5 
\\$a_k$ where $1\leq k \leq K$is the set of complex path gains(pg).
$N_1$ and $N_2$ are chosen so that $g_n$ is small when n is less than$-N_1$ and greater than $N_2$.In the given code,
\begin{align}
N_1=N_2=800
\end{align}
