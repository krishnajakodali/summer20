Rician fading channel is one of the useful models of real-world phenomena in wireless communication.These phenomena include multipath scattering effects , time dispesion and doppler shifts that arise from relative motion between transmitter and reciever.
\\
A channel filter applies path gains to the input signal.
Path gains are configured based on settings chosen in fading channel object or block which are used to model fading channels.
\\
The path gains are found using the matlab function ricianchan by running the code 
\begin{lstlisting}
./codes/LMS_equalizer_MATLAB.m
\end{lstlisting}
The channel specifications considered in the above code are as follows:
\begin{align}
t_s=\frac{1}{185000}
\\
fd=0.1
\\
k=0.87/0.13
\\
\tau=\myvec{0.0& 0.2& 0.4 &0.6 &0.8 }\times 10^{-5}
\\
pdb = \myvec{0& -2& -10 &-20 & -22}
\\
fdLos=0.7*fd=0.07
\end{align}
Where $t_s$ is the sample time of the input signal, fd is the maximum doppler shift in hertz, k is the rician K-factor in linear scale,fdLos is the doppler shift of line of sight component.
\\
$\tau$ is vector of path delays specified in seconds and pdb is the vector of average path gains specified in dB.

The path gains thus found are stored in the file
\begin{lstlisting}
./codes/path_gains.dat
\end{lstlisting}
